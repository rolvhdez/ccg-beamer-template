% ------------------------------
% Dummy code to present an Article
% ------------------------------

\section{Article section}

% Slide 1: Lists
\subsection{Lists}
\begin{frame}{Lists}
  \begin{itemize}
      \item Regular item
      \item \textbf{Bold item}
      \item \textit{Italic item}
  \end{itemize}
\end{frame}

% Slide 2: Columns
\subsection{Columns and images}

\begin{frame}{Columns and images}
\begin{columns}

    % Left Column
    \begin{column}{0.45\textwidth}

        \textbf{Bullet points}
        \begin{itemize}
            \item Item 1
            \item Item 2
            \item Item 3
            \item Referring Fig. \ref{fig:example-label}
        \end{itemize}

        \textbf{Numbered lists}
        \begin{enumerate}
          \item First item
          \item Second item
          \item Third item
        \end{enumerate}

        Maybe some citations here\footcite{anonymous2023website}.

    \end{column}

    % Right Column
    \begin{column}{0.45\textwidth}
      \begin{figure}
        \begin{center}
          \includegraphics[width=0.95\textwidth]{example-image-a}
        \end{center}
        \caption{Some cool caption\footcite{doe2023example}}
        \label{fig:example-label}
      \end{figure}
      
    \end{column}

\end{columns}
\end{frame}

% Slide 4: Equations
\begin{frame}{Equations}

  Here is an example of a numbered equation (Eq. \ref{eq:an-equation}):

  % Numbered equation
  \begin{equation}
    E = mc^2
    \caption{Even with caption}
    \label{eq:an-equation}
  \end{equation}

  And an inline equation: \( f(x) = x^2 + 2x + 1 \).

  A more complex equation, without numberation:

  % Numbered equation
  \begin{equation*}
    \int_{a}^{b} f(x) \, dx = F(b) - F(a)
  \end{equation*}
\end{frame}
