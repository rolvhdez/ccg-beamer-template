% ------------------------------
% Dummy code to present a Book
% ------------------------------

\section{Book section}

% Slide 5: Code Blocks
\subsection{Code blocks}

\begin{frame}[fragile] % Use [fragile] for code blocks

  Here is some example for \verb|Python|

  \frametitle{Code Blocks (Python)}
  \begin{lstlisting}[language=python]
    # A simple Python program
    name = "Alice"
    print(f"Hello, {name}!")
    numbers = [1, 2, 3, 4, 5]
    squared = [x**2 for x in numbers]
    print("Squared numbers:", squared)
  \end{lstlisting}

\end{frame}

\begin{frame}[fragile] % Use [fragile] for code blocks
  \frametitle{Code Blocks (R)}

  More than often we also code in \verb|R|

  \begin{lstlisting}[language=rlang]
    # A simple R program
    name <- "Alice"  # Define a variable
    print(paste("Hello,", name, "!"))  # Print a greeting
    numbers <- c(1, 2, 3, 4, 5)  # Create a vector
    squared <- numbers^2  # Square each number
    print("Squared numbers:")  # Print the result
    print(squared)
  \end{lstlisting}

\end{frame}

\begin{frame}[fragile] % Use [fragile] for code blocks
  \frametitle{Code Blocks (R)}

  Sometimes we also need to \verb|BASH| code

  \begin{lstlisting}[language=bash]
    #!/bin/bash
    
    # A simple Bash script
    echo "Hello, World!"  # Print a message
    name="Alice"  # Define a variable
    echo "Hello, $name!"  # Print a personalized message
    files=$(ls)  # List files in the current directory
    echo "Files in this directory: $files"  # Print the list
  \end{lstlisting}

\end{frame}

% Slide 6: Alert and Example Blocks
\subsection{Alert and Example Blocks}

\begin{frame}
  \frametitle{Alert and Example Blocks}
  \begin{block}{Regular Block}
    This is a regular block for general information.
  \end{block}

  \begin{alertblock}{Alert Block}
    This is an alert block to highlight something important!
  \end{alertblock}

  \begin{exampleblock}{Example Block}
    This is an example block to demonstrate something.
  \end{exampleblock}
\end{frame}

% Slide 7: Overlays (Step-by-Step Reveal)
\subsection{Overlays (Step-by-Step Reveal)}

\begin{frame}{Overlays (Step-by-Step Reveal)}

  \begin{columns}
    \begin{column}{0.45\textwidth}
      \begin{itemize}
          \item<1-> First item (appears on slide 1)
          \item<2-> Second item (appears on slide 2)
          \item<3-> Third item (appears on slide 3)
      \end{itemize}

      \uncover<4->{This text appears on slide 4.}
    \end{column}

    \begin{column}{0.45\textwidth}

      \begin{figure}
        \centering
        \includegraphics<5->[width=0.95\textwidth]{example-image-b}
        \uncover<5->{\caption{And this image will appear on slide 5.}}
        \label{fig:example-image-b}
      \end{figure}

    \end{column}
  \end{columns}

\end{frame}
